%\documentclass[12pt,a4paper]{report}
\documentclass[12pt,a4paper]{article}

\usepackage[brazil]{babel}
\usepackage[utf8]{inputenc}
\usepackage[T1]{fontenc}
\usepackage{graphicx,url}
\usepackage{hyperref}
%\usepackage{mathptmx}
\usepackage{lipsum}
\usepackage{booktabs}
\usepackage{pifont}
\usepackage{textcomp}
\usepackage{amsmath,amssymb}
\usepackage{listings}
\usepackage[scaled=0.8]{beramono}

\pdfgentounicode=1

\newcommand{\HRule}{\rule{\linewidth}{0.5mm}}

\begin{document}

\input{./cover.tex}

\section{Introdução}
Este relatório tem como objetivo apresentar os resultados obtidos a partir da
execução do primeiro exercício de programação paralela, que consiste na
paralelização do algoritmo \emph{count sort} utilizando a biblioteca OpenMP\@.
Primeiramente, o algoritmo desenvolvido e a estratégia de paralelização
utilizada é apresentada. Em seguida, é feita uma análise de desempenho
comparando os tempos de execução do algoritmo em diversas configurações, isto é,
variando-se a estratégia de escalonamento, o tamanho dos \textit{chunks} e o
número de \textit{threads}. O código completo deste trabalho está publicamente
disponível no
GitHub~\footnote{\url{https://github.com/alexandrelucchesi/parallel-programming-ex01}}.

\section{O Algoritmo}
O programa desenvolvido contempla três funcionalidades principais: geração de um
arquivo de dados contendo um número arbitrário de valores de ponto-flutuante,
processamento de um único arquivo de dados e a geração de um \textit{benchmark}.
Essas três funcionalidades são detalhadas nesta seção.

\subsection{Geração do Arquivo de Dados}
O algoritmo \textit{count sort} é um algoritmo de ordenamento que foi utilizado
neste trabalho para ordenar um vetor de números de ponto-flutuante. Dessa forma,
foi necessária a criação de funções para a geração de quantidades configuráveis
de números de ponto-flutante e que os gravasse em um formato apropriado para
servir de entrada para o programa, possibilitando a fácil experimentação do
\textit{count sort} com diferentes entradas de dados.

A interface dessas funções é apresentada a seguir:

\begin{verbatim}
void generate_floats(FILE* out, int count);
int read_input(FILE* fp, float** vector, int* size);
float* parse_floats(char* values_str, int count);
\end{verbatim}

A função \texttt{generate\_floats} recebe um arquivo de destino e a quantidade de
números que se deseja gerar. A função \texttt{read\_input} recebe um arquivo (no
formato gerado por \texttt{generate\_floats}) e retorna em \texttt{vector} e
\texttt{size} o vetor e seu tamanho, respectivamente. Por fim, a função
\texttt{parse\_floats} é utilizada internamente por \texttt{read\_input} para
realizar o \textit{parsing} de \textit{strings} para \textit{floats}.

Para gerar um arquivo de dados, \texttt{vec.dat}, contendo, por exemplo, 1000
elementos, basta executar o binário passando-se a \textit{flag} \texttt{-gen},
conforme descrito a seguir:

\begin{verbatim}
$ ./a.out -gen vec.dat 1000
\end{verbatim}

\subsection{Processamento}
O processamento de um arquivo de dados foi encapsulado na função
\texttt{process}, cuja assinatura é apresentada a seguir:

\begin{verbatim}
int process(float **vec, int *vec_size, double *time_sorting_only, double
        *time_with_input, const char *filename, omp_sched_t kind, int
        chunk_size);
\end{verbatim}

Os quatro primeiros parâmetros: \texttt{vec}, \texttt{vec\_size},
\texttt{time\_sorting\_only} e \texttt{time\_with\_input} são, na verdade, retornos
da função \texttt{process}, e retornam respectivamente, o vetor de
\textit{floats} ordenado, o tamanho do vetor, o tempo transcorrido considerando
apenas a execução do \textit{count sort} e o tempo transcorrido desde a entrada
de dados até o término da execução da função de ordenação. Os demais parâmetros
são utilizados para parametrizar a execução da função \textit{count\_sort} e
equivalem, respectivamente, ao nome do arquivo de dados, à política de
escalonamento do OpenMP (\textit{static}, \textit{dynamic}, \textit{guided} ou
\textit{auto}) e ao tamanho do \textit{chunk}.

Encapsular a lógica de processamento na função \textit{process} permitiu a
automatização do \textit{benchmarking} da aplicação, por meio do desenvolvimento
da função \texttt{bench} (descrita a seguir), evitando trabalho manual.

\subsection{\textit{Benchmark}}
 
\section{Resultados}


\section{Conclusão}
Por fim, é válido ressaltar que o programa está todo parametrizado via diretivas
de pré-processamento (\texttt{\#define}) e aceita alguns parâmetros em tempo de
execução (como o tamanho do \textit{chunk} e a política de escalonamento),
possibilitando a experimentação com diferentes configurações. O \textit{design}
escolhido permite variar de forma fácil a política de escalonamento, o tamanho
dos \textit{chunks} e a quantidade de \textit{threads} de forma não intrusiva e
modular.

%% Booktabs require to add \usepackage{booktabs} to your document preamble
%\begin{table}[h]
%\centering
%\caption{Cronograma de Execução}
%\begin{tabular}{@{}ccccccc@{}}
%\toprule
%Etapa & Semana 1   & Semana 2   & Semana 3   & Semana 4   & Semana 5   & Semana 6      \\ \midrule
%1     & \ding{117} & \ding{117} & \ding{117} &            &            & \\ \midrule
%2     &            &            & \ding{117} & \ding{117} &            & \\ \midrule
%3     &            &            &            &            & \ding{117} & \\ \midrule
%4     &            &            &            &            & \ding{117} & \ding{117}    \\ \midrule \\ \bottomrule
%\end{tabular}
%\end{table}

\bibliographystyle{plain}
\bibliography{references.bib}

\end{document}








